\documentclass{article}
\usepackage{kotex}


\begin{document}

\title{\textbf{OSS 팀 프로젝트 - pacmen}}
\author {20134827 박진범\\20134821 최석원\\20134800 이진호}
\date{2017년 12월 8일}
\maketitle



\newpage
\textbf{목차}\\\\
\begin{enumerate}
\item 오픈 소스 SW 선택의 이유 및 개요 \\
\item 선택한 오픈 소스 SW 실행 방법 \\
\item 오픈 소스 SW 사용 설명서\\
\item 기능 향상을 위한 제안\\
\item 느낀점
\end{enumerate}

\newpage
12월 3일(일)\\\\

- github 사용법 및 Latex 설치, 사용법 익힘\\

- 오픈소스 주제 고민\\

- 가계부, 오피스, 게임 중 pacman 게임으로 선정\\

- 내일 만나서 오픈소스 실행 해볼 예정\\


12월 4일(월)\\\\

- 어제 선택한 pacman 오픈소스 다운로드 및 설치 환경 구축\\

- 여러 환경(windows, ubuntu) 중 ubuntu를 이용하여 실행\\

- 게임 실행 및 방법 확인\\

- 본격적으로 사용 설명서 및 선택 이유 작성 시작\\

\flushleft
\textbf{1. - 오픈소스 선택 이유 -}
\begin{itemize}

\item 첫 번째로 여러가지 오픈 소스 범주 중 게임을 선택한 이유는 플래시 게임, 온라인 게임, 모바일 게임까지 한번쯤은 다 접해봤기 때문에 쉽게 접근 할 수 있을거라고 생각을 해서 게임을 선택했습니다. 또한 설명서를 작성하면서 잠깐이나마 즐기면서 작성할 수 있다는 점을 고려해 최종적으로 게임을 선택하였습니다.\\

\item 여러 게임이 있지만 pacman을 선택한 이유는 어릴 적 플래시로 플레이 했던 기억이 있고, 소스나 개발 환경을 현실적인 우리의 실력을 바탕으로 구성할 수 있을 정도의 오픈 소스 게임을 찾다 보니 최종적으로 pacman을 선택하게 되었습니다.

\end{itemize}

\newpage
\textbf{2. - 오픈 소스 SW 실행 방법 (ubuntu) -}
\begin{enumerate}
\item Oracle VM Virtual Machine을 실행합니다. \\
설치 : http://pentode.tistory.com/79\\

\item Vircual Machine에 설치 되어있는 Ubuntu를 실행합니다. \\
설치 : https://extrememanual.net/7223\\

\item https://github.com 에 접속 후 pacman의 자료들(pacman-master)을 받은 후 terminal을 실행합니다. \\※ 설치 경로 : 바탕화면으로 가정\\
\item ls를 입력 후 현재 위치를 인지한 후에 cd 바탕화면 명령어를 입력해서 바탕화면으로 이동한 후 cd pacman-master명령어를 입력하여 pacman-master폴더의 위치로 이동합니다.\\
\item 다시 ls를 입력하면 pacman-master폴더 안에있는 파일들이 나오게 되는데 make 명령어를 입력합니다. ( make 명령어 입력 후 출력되는 메시지 ) (캡쳐본 활용)\\
\item pacman을 입력하면 아래와 같은 메시지가 나올것입니다. \\
The program 'pacman' is currently not installed. You can install it by typing: sudo apt install pacman\\
\item sudo apt install pacman 입력하라고 메시지가 출력됩니다. 입력하면 패스워드 입력하라고 메시지가 출력되는데 현재 사용하고있는 사용자 pc의 암호를 입력합니다. 입력하면 아래와 같은 메시지가 나옵니다 (캡쳐본 활용)
\end{enumerate}

\newpage
\textbf{3. - 오픈 소스 SW 사용 설명서 -}
\begin{itemize}
\item pacman 또는 packman.out으로 만들어진 파일을 실행시킨다.(실행시 첫 시작화면)
\item 조작법 : 키보드의 ↑(위), ↓(아래), →(오른쪽), ←(왼쪽) 버튼을 이용한다.
\item 시작은 맵의 중하단부의 노란색 팩맨으로 부터 시작하며 유령을 피해 도망다녀야 합니다.
\item 유령은 빨간색, 주황색, 자주색, 분홍색 총 4마리로 잡히면 목숨이 1개 줄어듭니다.
\item 목숨은 시작시 총 3개로 시작하며 유령에 부딪혀 죽을때마다 1씩 감소하고 0이될시 게임오버가 됩니다.
\item 방향키를 이용 유령을 피해 돌아다니면서 노란색으로 된 작은 점을 제한시간(bonus)동안 최대한 많이 먹어야합니다. 점하나당 +10점
\item 맵의 4방향 끝에 위치한 노란색 큰점을 먹으면 Supertime을 가지게 됩니다.
\item Supertime은 약 7초로 팩맨의 이동속도가 빨라지고 회색으로 변한 유령들을 잡아먹을 수 있으며 잡을시 유령은 가운데의 감옥으로 들어갑니다. 유령당 +100점
\item 잡아먹은 유령은 잡힌 시점부터 약 7초 후 다시 탈출합니다.
\item 유령들을 피해 노란점들을 전부다 먹으면 라운드가 클리어 되고 즉시 똑같은 맵으로 된 다음 라운드가 시작됩니다.
\item 점수는 클리어했을때 이어지며 Hiscore에 점수가 기록되고 Level이 1에서\\ 2로 증가합니다.
\item 목숨 내에 가장 높은 점수를 기록한 사람이 이기게 됩니다.
\end{itemize}


\end{document}